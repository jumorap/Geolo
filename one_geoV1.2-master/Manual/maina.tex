
%----------------------------------------------------------------------------------------
%	PACKAGES AND OTHER DOCUMENT CONFIGURATIONS
%----------------------------------------------------------------------------------------

\documentclass{tufte-book} % Use the tufte-book class which in turn uses the tufte-common class
\usepackage[spanish]{babel}
\usepackage[latin1]{inputenc}
\usepackage{flowfram}		
\hypersetup{colorlinks} % Comment this line if you don't wish to have colored links
\usepackage{microtype} % Improves character and word spacing
\usepackage{lipsum} % Inserts dummy text
\usepackage{booktabs} % Better horizontal rules in tables
\usepackage{graphicx}										% figures
\usepackage{url}											% URLs
\usepackage[usenames,dvipsnames]{xcolor}					% olor
\usepackage{multicol}										% columns env.
\setlength{\multicolsep}{0pt}
\usepackage{paralist}										% compact lists
\usepackage{tikz}
\usepackage{mfirstuc}
\usepackage{lipsum}
\usepackage{anyfontsize}
\usepackage{nopageno}
\usepackage{titlesec}
\usepackage{hyperref}
\usepackage{soul}
\usepackage{anyfontsize}
\graphicspath{{graphics/}} % Sets the default location of pictures
\setkeys{Gin}{width=\linewidth,totalheight=\textheight,keepaspectratio} % Improves figure scaling
\usepackage{fancyvrb} % Allows customization of verbatim environments
\fvset{fontsize=\normalsize} % The font size of all verbatim text can be changed here
\newcommand{\hangp}[1]{\makebox[0pt][r]{(}#1\makebox[0pt][l]{)}} % New command to create parentheses around text in tables which take up no horizontal space - this improves column spacing
\newcommand{\hangstar}{\makebox[0pt][l]{*}} % New command to create asterisks in tables which take up no horizontal space - this improves column spacing
\usepackage{xspace} % Used for printing a trailing space better than using a tilde (~) using the \xspace command
\newcommand{\monthyear}{\ifcase\month\or Enero\or Febrero\or Marzo\or Abril\or Mayo\or Junio\or Julio\or Augosto\or Septiembre\or Octubre\or Noviembre\or Diciembre\fi\space\number\year} % A command to print the current month and year


\usepackage[T1]{fontenc}
\textsl{}\usepackage[math]{iwona}
%\titleformat*{\chapter}{\fontsize{24}{24}\selectfont \bfseries}
%\titleformat*{\section}{\fontsize{22}{22}\selectfont \bfseries}
%\titleformat*{\subsection}{\large \selectfont}
%

\renewcommand*\rmdefault{iwona}

\newcommand{\openepigraph}[2]{ % This block sets up a command for printing an epigraph with 2 arguments - the quote and the author
\begin{fullwidth}
\sffamily\large
\begin{doublespace}
\noindent\allcaps{#1}\\ % The quote
\noindent\allcaps{#2} % The author
\end{doublespace}
\end{fullwidth}
}

\newcommand{\blankpage}{\newpage\hbox{}\thispagestyle{empty}\newpage} % Command to insert a blank page

\usepackage{units} % Used for printing standard units

\newcommand{\hlred}[1]{\textcolor{Maroon}{#1}} % Print text in maroon
\newcommand{\hangleft}[1]{\makebox[0pt][r]{#1}} % Used for printing commands in the index, moves the slash left so the command name aligns with the rest of the text in the index 
\newcommand{\hairsp}{\hspace{1pt}} % Command to print a very short space
\newcommand{\ie}{\textit{i.\hairsp{}e.}\xspace} % Command to print i.e.
\newcommand{\eg}{\textit{e.\hairsp{}g.}\xspace} % Command to print e.g.
\newcommand{\na}{\quad--} % Used in tables for N/A cells
\newcommand{\measure}[3]{#1/#2$\times$\unit[#3]{pc}} % Typesets the font size, leading, and measure in the form of: 10/12x26 pc.
\newcommand{\tuftebs}{\symbol{'134}} % Command to print a backslash in tt type in OT1/T1

\providecommand{\XeLaTeX}{X\lower.5ex\hbox{\kern-0.15em\reflectbox{E}}\kern-0.1em\LaTeX}
\newcommand{\tXeLaTeX}{\XeLaTeX\index{XeLaTeX@\protect\XeLaTeX}} % Command to print the XeLaTeX logo while simultaneously adding the position to the index

\newcommand{\doccmdnoindex}[2][]{\texttt{\tuftebs#2}} % Command to print a command in texttt with a backslash of tt type without inserting the command into the index

\newcommand{\doccmddef}[2][]{\hlred{\texttt{\tuftebs#2}}\label{cmd:#2}\ifthenelse{\isempty{#1}} % Command to define a command in red and add it to the index
{ % If no package is specified, add the command to the index
\index{#2 command@\protect\hangleft{\texttt{\tuftebs}}\texttt{#2}}% Command name
}
{ % If a package is also specified as a second argument, add the command and package to the index
\index{#2 command@\protect\hangleft{\texttt{\tuftebs}}\texttt{#2} (\texttt{#1} package)}% Command name
\index{#1 package@\texttt{#1} package}\index{packages!#1@\texttt{#1}}% Package name
}}

\newcommand{\doccmd}[2][]{% Command to define a command and add it to the index
\texttt{\tuftebs#2}%
\ifthenelse{\isempty{#1}}% If no package is specified, add the command to the index
{%
\index{#2 command@\protect\hangleft{\texttt{\tuftebs}}\texttt{#2}}% Command name
}
{%
\index{#2 command@\protect\hangleft{\texttt{\tuftebs}}\texttt{#2} (\texttt{#1} package)}% Command name
\index{#1 package@\texttt{#1} package}\index{packages!#1@\texttt{#1}}% Package name
}}

% A bunch of new commands to print commands, arguments, environments, classes, etc within the text using the correct formatting
\newcommand{\docopt}[1]{\ensuremath{\langle}\textrm{\textit{#1}}\ensuremath{\rangle}}
\newcommand{\docarg}[1]{\textrm{\textit{#1}}}
\newenvironment{docspec}{\begin{quotation}\ttfamily\parskip0pt\parindent0pt\ignorespaces}{\end{quotation}}
\newcommand{\docenv}[1]{\texttt{#1}\index{#1 environment@\texttt{#1} environment}\index{environments!#1@\texttt{#1}}}
\newcommand{\docenvdef}[1]{\hlred{\texttt{#1}}\label{env:#1}\index{#1 environment@\texttt{#1} environment}\index{environments!#1@\texttt{#1}}}
\newcommand{\docpkg}[1]{\texttt{#1}\index{#1 package@\texttt{#1} package}\index{packages!#1@\texttt{#1}}}
\newcommand{\doccls}[1]{\texttt{#1}}
\newcommand{\docclsopt}[1]{\texttt{#1}\index{#1 class option@\texttt{#1} class option}\index{class options!#1@\texttt{#1}}}
\newcommand{\docclsoptdef}[1]{\hlred{\texttt{#1}}\label{clsopt:#1}\index{#1 class option@\texttt{#1} class option}\index{class options!#1@\texttt{#1}}}
\newcommand{\docmsg}[2]{\bigskip\begin{fullwidth}\noindent\ttfamily#1\end{fullwidth}\medskip\par\noindent#2}
\newcommand{\docfilehook}[2]{\texttt{#1}\index{file hooks!#2}\index{#1@\texttt{#1}}}
\newcommand{\doccounter}[1]{\texttt{#1}\index{#1 counter@\texttt{#1} counter}}

\usepackage{makeidx} % Used to generate the index
\makeindex % Generate the index which is printed at the end of the document

% This block contains a number of shortcuts used throughout the book

\newcommand{\GM}{\textit{Gmas $+$}\xspace}
\newcommand{\OG}{\textit{One Geo}\xspace}
\newcommand{\SDs}{\textit{secciones delgadas}\xspace}
\newcommand{\SD}{\textit{secci�n delgada}\xspace}


%----------------------------------------------------------------------------------------
%	BOOK META-INFORMATION
%----------------------------------------------------------------------------------------

\title{Manual de Operacion de OneGeo} % Title of the book

\author[Ing. Fredy Osorio]{Ing. Fredy Osorio\textbackslash Gmas} % Author

%\publisher{Publisher of This Book} % Publisher

%----------------------------------------------------------------------------------------

\begin{document}

\frontmatter

%----------------------------------------------------------------------------------------
%	EPIGRAPH
%----------------------------------------------------------------------------------------

\thispagestyle{empty}
%\openepigraph{The public is more familiar with bad design than good design. It is, in effect, conditioned to prefer bad design, because that is what it lives with. The new becomes threatening, the old reassuring.}{Paul Rand, {\itshape Design, Form, and Chaos}}
%\vfill
%\openepigraph{A designer knows that he has achieved perfection not when there is nothing left to add, but when there is nothing left to take away.}{Antoine de Saint-Exup\'{e}ry}
%\vfill
%\openepigraph{\ldots the designer of a new system must not only be the implementor and the first large-scale user; the designer should also write the first user manual\ldots If I had not participated fully in all these activities, literally hundreds of improvements would never have been made, because I would never have thought of them or perceived why they were important.}{Donald E. Knuth}

%----------------------------------------------------------------------------------------

\maketitle % Print the title page

%----------------------------------------------------------------------------------------
%	COPYRIGHT PAGE
%----------------------------------------------------------------------------------------

\newpage
\begin{fullwidth}
~\vfill
\thispagestyle{empty}
\setlength{\parindent}{0pt}
\setlength{\parskip}{\baselineskip}
Copyright \copyright\ \the\year\ \thanklessauthor

%\par\smallcaps{Published by \thanklesspublisher}

%\par\smallcaps{tufte-latex.googlecode.com}

%\par Licensed under the Apache License, Version 2.0 (the ``License''); you may not use this file except in compliance with the License. You may obtain a copy of the License at \url{http://www.apache.org/licenses/LICENSE-2.0}. Unless required by applicable law or agreed to in writing, software distributed under the License is distributed on an \smallcaps{``AS IS'' BASIS, WITHOUT WARRANTIES OR CONDITIONS OF ANY KIND}, either express or implied. See the License for the specific language governing permissions and limitations under the License.\index{license}

\par\textit{Primera impresi�n, \monthyear}
\end{fullwidth}

%----------------------------------------------------------------------------------------

\tableofcontents % Print the table of contents

%----------------------------------------------------------------------------------------

\listoffigures % Print a list of figures

%----------------------------------------------------------------------------------------

\listoftables % Print a list of tables


%----------------------------------------------------------------------------------------
%	INTRODUCTION
%----------------------------------------------------------------------------------------

\cleardoublepage

\chapter{Introducci�n}
\label{ch:intro}

Este documento tiene como prop�sito proporcionar una gu�a detallada para llevar a cabo la operaci�n cotidiana del microscopio de \OG de \GM, adem�s de ofrecer una manera ra�da y eficiente de desempe�ar las tareas de toma y procesamiento de \SDs y de esta manera apoyar las tareas en el laboratorio de geolog�a y petrof�sica de la compa��a.

\section{Objetivo del Manual de Operaci�n}

El Objetivo del presente manual es el de proveer una gu�a y procedimientos de operaci�n del microscopio de \OG que permita la efectiva y �gil toma y procesamiento de muestras de \SDs.

\section{Alcance del Manual}

El alcance de presente documento abarca las actividades a realizar desde la recepci�n de la \SD hasta el procedimiento de procesamiento de im�genes (\textit{stitching}) e ingreso del resultado a la plataforma web de \OG.
%----------------------------------------------------------------------------------------
%	BIBLIOGRAPHY
%----------------------------------------------------------------------------------------

%\bibliography{bibliography} % Use the bibliography.bib file for the bibliography
%\bibliographystyle{plainnat} % Use the plainnat style of referencing

%----------------------------------------------------------------------------------------

%\printindex % Print the index at the very end of the document

\end{document}